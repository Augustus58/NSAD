\documentclass[conference]{IEEEtran}
\IEEEoverridecommandlockouts
% The preceding line is only needed to identify funding in the first footnote. If that is unneeded, please comment it out.
\usepackage{cite}
\usepackage{amsmath,amssymb,amsfonts}
\usepackage{algorithmic}
\usepackage{graphicx}
\usepackage{textcomp}
\def\BibTeX{{\rm B\kern-.05em{\sc i\kern-.025em b}\kern-.08em
    T\kern-.1667em\lower.7ex\hbox{E}\kern-.125emX}}

\begin{document}

\title{Xamarin.Forms Development Fundamentals}

\author{
  \IEEEauthorblockN{
    Aku Silvennoinen
  }
  \IEEEauthorblockA{
    \textit{Department of Computer Science} \\
    \textit{University of Helsinki}\\
    Helsinki, Finland \\
    aku.silvennoinen@helsinki.fi
  }
}

\maketitle

\begin{abstract}
  todo
\end{abstract}

\begin{IEEEkeywords}
  Xamarin.Forms, MVVM
\end{IEEEkeywords}

\section{Introduction}

Targeting multiple mobile platforms in software development is a challenge because of different user-interface paradigms, different development environments, different programming interfaces, and different programming languages. At the moment, the most relevant mobile platforms are Android and iOS. One of the main differences between user-interface paradigms for Android and iOS is the concept of back button. In Android, there is back button at the right bottom corner, no matter which view is active. In iOS there is no such button but going back can happen by swiping the screen, pressing a button in an user interface or pressing the home button. Android development happens usually in Android Studio~\cite{androidstudio} environment whereas iOS development happens in Xcode environment~\cite{xcode}. Programming interfaces are different. For example toggling a Boolean value in Android happens by using a ``widget'' called \texttt{Switch} whereas in iOS it happens by using a ``view'' called \texttt{UISwitch}. Android development is done traditionally with Java programming language whereas iOS development is done traditionally in Objective-C.

A trivial, but laborious strategy is to just develop distinct versions of an application for all platforms. An advantage of this strategy is that it is much easier to find developers for native Android and iOS development. A disadvantage is that an application should be developed as many times as there are target platforms. A lot of designing work can be merged though.

Another strategy is the cross platform development, where possibly all lines of written code are shared between target platforms. Solutions available for cross platform mobile development are for example 
Apache~Cordova~\cite{cordova}, React~Native~\cite{react}, and Xamarin.Forms~\cite{xamarin.forms}. The last is discussed in this paper.

Xamarin is a San Francisco -based company founded in 2011 and owned by Microsoft. Xamarins main product is Xamarin Platform and Xamarin.Forms is an part of it. Xamarin is founded by the engineers who created Mono, Mono for Android and MonoTouch. Mono is a cross platform .NET~Framework~\cite{mono}. ``.NET is a free, cross-platform, open source developer platform for building many different types of applications.''~\cite{dotnet}. Mono for Android is the ancestor of Xamarin.Android -- a .NET library which implements the Android API. Finally, MonoTouch is the ancestor of Xamarin.iOS -- a .NET library which implements the iOS API.

Xamarin.Forms uses C\#~\cite{csharp} as the programming language, Visual~Studio~\cite{vs} or Visual~Studio~for~Mac as the development environment, and .NET~Framework class library for core functionalities, such as networking and JSON reading and writing. C\# is a multi-paradigm programming language which is closely associated to the .NET~Framework. C\# have had language level support right from the beginning for properties~\cite{properties} and events~\cite{events} which are useful in programming graphical user interfaces. C\# supports also asynchronous programming~\cite{asyncro}.

Xamarin.Forms uses the MVVM-architecture (Model-View-ViewModel) for dividing views and business logic. The Model handles the underlying data. The View handles the user interface. Finally, the ViewModel handles interaction between the Model and the View.

\subsection*{Research Goals}

The main research goal of this paper is to answer the question:

\emph{What should a developer know about Xamarin.Forms in order to do professional mobile development with it?}

\nocite{xamarin}
\bibliographystyle{./IEEEtran}
\bibliography{bibliography}

\end{document}
